\documentclass[a4paper, 12pt]{article}
\usepackage[slovene]{babel}
\usepackage[utf8]{inputenc}
\usepackage{url}
\usepackage{hyperref}
\usepackage{eurosym}

\begin{document}

\title{Najemnine v Ljubljani}
\author{Domen Mohorčič}
\maketitle

\section{Uvod}

Ko se povprečen Slovenec odseli od staršev v svoje stanovanje, je star 28,2
leti (\href{https://www.stat.si/statweb/News/Index/7570}{stat.si}). Pred
izselitvijo pa si mora stanovanje poiskati. Po navadi ljudje pri izbiri
stanovanja gledajo na to, ali jim je stanovanje všeč in ali se jim zdi cena
ustrezna stanovanju. Od česa pa sploh je odvisna cena stanovanja? Friškovec
(2010)[1] ugotavlja, da je oglaševalska cena stanovanja
pozitivno odvisna od površine, števila kopalnic in ali gre za mansardno
stanovanje, negativno pa predvsem od višine nadstropja. Repič (2014)[2] pa je
v magistrski nalogi pokazala, da je prodajna cena stanovanja pozitivno odvisna
od prisotnosti dvigala, parkirnega mesta, opremljenosti stanovanja, bližine
središča Ljubljane in števila sob v stanovanju, negativno pa na ceno vplivajo
starost stanovanja, površina in trajanje, ko je nepremičnina na voljo za
prodajo.

V Sloveniji v najemniških stanovanjih živi le 2,4
gospodinjstev (\href{https://cekin.si/nepremicnine/trnova-pot-do-lastnega-doma-zakaj-mora-biti-tako.html}{cekin.si}).
Kljub temu je trg najema nepremičnin kar velik, še posebej v Ljubljani (na
nepremicnine.net je od 1500 oglasov za najem, od tega kar 1000 v Ljubljani),
kjer pa so najbolj zaželjeni študentje ali posamezniki. Nikjer pa nisem
zasledil raziskave, ki bi ugotavljala, kaj vpliva na ceno najema.

Namen seminarske naloge je ugotoviti, kateri dejavniki najbolj vplivajo na ceno
najemnin v Ljubljanskima predeloma Vič in Rudnik.

\section{Podatki}

Podatke sem pridobil iz slovenske spletne strani
\href{https://www.nepremicnine.net}{nepremicnine.net} dne 8.8.2020.
Iskal sem stanovanja v Ljubljani v predelih Vič in Rudnik.
Pri pregledovanju oglasov sem se osredotočil na naslednje spremenljivke:
nadstropje, število vseh nadstropij, leto gradnje, leto prenove, število sob,
ali ima stanovanje shrambo/klet, ali je stanovanje opremljeno, število
pripadajočih parkirišč, površina, zunanje površine (balkon, vrt, \dots),
mesečni stroški in cena najema. Ker pa sem hotel ugotoviti, ali
na ceno najema vpliva tudi lokacija stanovanja, sem poiskal še oddaljenost
do središča Ljubljane (v mojem primeru Prešernov trg). Na prej omenjeni spletni
strani pa v večini primerov ni napisanega točnega naslova, zato sem iskal samo
približne lokacije (ulica ali naselje). Pri določanju razdalje sem si pomagal
z orodjem \href{https://www.distance.to/}{distance.to}.

\subsection{Opis podatkov}

Zvezne spremenljivke so naslednje:
\begin{center}
\begin{tabular}{ c|c|c}
	spremenljivka & opis & enota \\
	\hline
	\hline
	nadstropje & V katerem nadstropju se stanovanje nahaja & - \\
	\hline
	vsaNadstropja & Število vseh nadstropij v stavbi & - \\
	\hline
	letoGradnje & Leto, v katerem je bilo stanovanje zgrajeno & - \\
	\hline
	letoPrenove & Leto, v katerem je bilo stanovanje prenovljeno & - \\
	\hline
	stSob & Število sob v stanovanju & - \\
	\hline
	stParkirisc & Število parkirnih mest, ki pripadajo stanovanju & - \\
	\hline
	zunanjePovrsine & Vsota zunanjih površin stanovanja & $ m^{2} $ \\
	\hline
	povrsina & Velikost bivalne površine v stanovanju & $ m^{2} $ \\
	\hline
	oddaljenost & Oddaljenost stanovanja od Prešernovega trga & km \\
	\hline
	cena & Cena mesečne najemnine stanovanja & \euro \\
	\hline
	stroski & Cena mesečnih stroškov bivanja & \euro \\
\end{tabular}
\end{center}

\begin{center}
\begin{tabular}{ c|ccccccc }
	spremenljivka & min & 1. kvartil & mediana & 3. kvartil & max & povpr. \\
	\hline
	nadstropje & 0 & 0 & 1 & 2 & 7 & 1,505 \\
	vsaNadstropja & 1 & 2 & 2 & 3,5 & 9 & 2,689 \\
	letoGradnje & 1895 & 1970 & 1995 & 2006 & 2020 & 1986 \\
	stSob & 1 & 1 & 2 & 3 & 4 & 2,228 \\
	stParkirisc & 0 & 0 & 1 & 1 & 5 & 0,796 \\
	zunanjePovrsine & 0 & 0 & 4,5 & 14,5 & 600 & 34,690 \\
	povrsina & 10 & 38,6 & 65 & 91,5 & 150 & 67,75 \\
	oddaljenost & 0,97 & 1,855 & 2,18 & 3,12 & 5,07 & 2,53 \\
	cena & 130 & 515 & 750 & 950 & 3100 & 879,6 \\
	stroski & 0 & 0 & 0 & 100 & 600 & 60,49 \\
\end{tabular}
\end{center}

Diskretne spremenljivke
\begin{center}
\begin{tabular}{ c|c c }
	 & da & ne \\
	\hline
	shramba & 51 & 52 \\
	\hline
	 & 49,5\% & 50,5\% \\
\end{tabular}
\end{center}
\begin{center}
\begin{tabular}{ c|c c c }
	 & da & delno & ne \\
	\hline
	opremljenost & 92 & 8 & 3 \\
	\hline
	 & 89,3\% & 7,8\% & 2,9\% \\
\end{tabular}
\end{center}

\section{Rezultati}

\section{Komentar}

\section{Literatura}

[0]
https://www.distance.to/

[1] Analiza dejavnikov oglaševanih cen rabljenih stanovanj v Ljubljani in njeni
okolici, Sonja Friškovec, Aleksander Janeš, Univerza na Primorskem, jesen 2010

[2] Dejavniki oblikovanja prodajnih cen stanovanj, Mojca Repič, Ljubljana,
oktober 2014

\end{document}