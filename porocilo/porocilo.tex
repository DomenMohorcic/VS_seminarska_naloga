\documentclass[a4paper, 11pt]{article}
\usepackage[slovene]{babel}
\usepackage[utf8]{inputenc}
\usepackage{url}
\usepackage{hyperref}

\begin{document}

\title{Najemnine v Ljubljani}
\author{Domen Mohorčič}
\maketitle

\section{Uvod}

\section{Podatki}

Podatke sem pridobil iz slovenske spletne strani
\href{https://www.nepremicnine.net}{nepremicnine.net} dne 8.8.2020.
Iskal sem stanovanja v Ljubljani v predelih Vič in Rudnik.
Pri pregledovanju oglasov sem se osredotočil na naslednje spremenljivke:
nadstropje, število vseh nadstropij, leto gradnje, leto prenove, število sob,
ali ima stanovanje shrambo/klet, ali je stanovanje opremljeno, število
pripadajočih parkirišč, površina, zunanje površine (balkon, vrt, \dots),
mesečni stroški in cena najema. Ker pa sem hotel ugotoviti, ali
na ceno najema vpliva tudi lokacija stanovanja, sem poiskal še oddaljenost
do središča Ljubljane (v mojem primeru Prešernov trg). Na prej omenjeni spletni
strani pa v večini primerov ni napisanega točnega naslova, zato sem iskal samo
približne lokacije (ulica ali naselje). Pri določanju razdalje sem si pomagal
z orodjem \href{https://www.distance.to/}{distance.to}.

\subsection{Opis podatkov}

Zvezne spremenljivke
\begin{center}
\begin{tabular}{ c|c c c c c c c }
	spremenljivka & min & 1. kvartil & mediana & 3. kvartil & max & povpr. \\
	\hline
	nadstropje & 0 & 0 & 1 & 2 & 7 & 1.505 \\
	vsaNadstropja & 1 & 2 & 2 & 3.5 & 9 & 2.689 \\
	letoGradnje & 1895 & 1970 & 1995 & 2006 & 2020 & 1986 \\
	stSob & 1 & 1 & 2 & 3 & 4 & 2.228 \\
	stParkirisc & 0 & 0 & 1 & 1 & 5 & 0.796 \\
	zunanjePovrsine & 0 & 0 & 4.5 & 14.5 & 600 & 34.690 \\
	povrsina & 10 & 38.6 & 65 & 91.5 & 150 & 67.75 \\
	oddaljenost & 0.97 & 1.855 & 2.18 & 3.12 & 5.07 & 2.53 \\
	cena & 130 & 515 & 750 & 950 & 3100 & 879.6 \\
	stroski & 0 & 0 & 0 & 100 & 600 & 60.49 \\
\end{tabular}
\end{center}

Diskretne spremenljivke
\begin{center}
\begin{tabular}{ c|c c }
	 & da & ne \\
	\hline
	shramba & 51 & 52 \\
	\hline
	 & 49.5\% & 50.5\% \\
\end{tabular}
\end{center}
\begin{center}
\begin{tabular}{ c|c c c }
	 & da & delno & ne \\
	\hline
	opremljenost & 92 & 8 & 3 \\
	\hline
	 & 89.3\% & 7.8\% & 2.9\% \\
\end{tabular}
\end{center}

\section{Rezultati}

\section{Komentar}

\section{Literatura}

https://www.distance.to/

\end{document}